\documentclass[12pt,a4paper]{amsart}
\usepackage{a4wide}
\usepackage[utf8]{inputenc}
\usepackage[T2A]{fontenc}
\usepackage{graphics,graphicx,epsfig}
\usepackage{amssymb,amsfonts,amsthm,amsmath,mathtext,cite,enumerate,float}
\usepackage[english,russian]{babel}
\usepackage[all]{xy}
\usepackage{morefloats}
\usepackage{pgf}
\usepackage{tikz}

% Comment the following block when compiling this .tex with a saner compiler than texlive.
\makeatletter
\def\@settitle{\begin{center}%
    \baselineskip14\p@\relax
    \bfseries
    \@title
  \end{center}%
}
\makeatother

\begin{document}
% Comment the following block when compiling this .tex with a saner compiler than texlive.
\pagestyle{plain}

\title{Выбор функции активации при прогнозировании нейронными сетями}
\author{Г.\,И.~Рудой}
\address{Московский физико-технический институт, ФУПМ, каф. <<Интеллектуальные системы>>}
\thanks{Научный руководитель В.\,В.~Стрижов}

\begin{abstract}
  Черновик работы по порождению существенно нелинейных авторегрессионных моделей.
\end{abstract}

\maketitle

\section{Система обозначений}

% TODO rewrite with the glossaries package ( http://en.wikibooks.org/wiki/LaTeX/Glossary )
\begin{itemize}
  \item $ \mathcal{A} $ --- множество индексов информативных признаков.
  \item $ G $ --- множество элементарных функций, используемых для составления суперпозиций.\footnote{Имеет смысл четко определить, откуда и куда действуют функции из $G$: $G = { g : R^n \to R^n \mid \forall n }$ или же $ G = { g : R \to R \lor R \times R \to R} $. В принципе, второго случая должно за глаза хватить --- используемые функции либо одноместные (например, $\log$, $\tan$, $\neg$), либо двухместные (например, $+$, $\times$, $\land$).}
  \item $ \overline{G} $ --- пополненное множество элементарных функций: $ \overline{G} = G\ \cup\ G_X = \{ \overline{g}_i : X \to R \mid i = 1, \dots, N\} $, где $\overline{g}_i$ --- функция, возвращающая $i$-ый признак из множества признаков $X$.
  \item $ \mathcal{I} $ --- множество индексов всех признаков.
  \item $ N $ --- мощность множества числовых признаков (число признаков).
  \item $ \overline{N} $ --- мощность пополненного множества числовых признаков ($\overline{N} = |G| + N$).
  \item $ x_i $ --- признак.
  \item $ X $ --- множество числовых признаков: $ X = \cup_{i = 1}^N {x_i}$.
\end{itemize}

\section{Описание регрессионных моделей при помощи матриц}

Условимся руководствоваться следующим очевидным правилом при составлении суперпозиций: в листьях соответствующего дерева выражений содержатся функции из $G_X$ и только они.

Сопоставим суперпозиции матрицу, ее описывающую, и укажем способ ее построения.

Пусть дана произвольная суперпозиция функций из $G$ в инфиксной записи. Требуется построить соответствующую данной суперпозиции матрицу. Таким образом мы заодно и опишем, что представляет из себя матрица, характеризующая суперпозицию, и укажем некоторые ее свойства.

Первым шагом является построение дерева грамматического разбора выражения в инфиксной записи, эта тема хорошо изучена и разобрана, например, в \cite{Aho86}, поэтому положим, что мы уже имеем дерево выражения. Заметим, что дерево выражения является ориентированным графом. Пронумеруем его вершины, запустив по графу поиск в глубину. Присвоим каждой вершине номер, соответствующий ее порядку при таком обходе в глубину, и условимся считать первой вершину, из которой был начат обход.

Если в суперпозиции участвует $n$ функций, то, соответственно, получим индексы $1, \dots, n$. Построим матрицу $A = \| a_{ij} \|$ размера $n \times n$ и запишем в элемент $a_{ij}$ единицу, если в графе есть ребро, соединяющее $i$-ую и $j$-ую вершины, иначе --- ноль.

Так как в $G$ могут существовать коммутативные сами с собой операции, то порядок следования узлов графа, вообще говоря, недетерминирован, поэтому вместе с матрицей необходимо хранить вектор размерности $n$ 

Заметим следующие очевидные свойства:

\begin{itemize}
  \item На диагонали матрицы стоят нули (так как в графе по определению отсутствуют петли).
  \item Матрица верхнетреугольная (так как граф --- дерево, и по построению нумерации ребра есть только от элементов с меньшим номером к элементам с большим).
  \item В каждом столбце может быть только одна единица, исключая первый столбец, в котором ноль единиц.
  \item Сумма числа нулей в строке характеризует арность функции с соответствующим индексом и ее принадлежность к $G$ либо $G_X$: унарная функция из $G$ имеет одну единицу, бинарная $G$ --- две, а имеющая только нули функция лежит в $G_X$.
\end{itemize}

\subsection{Сложность модели}

Если считать сложностью модели число используемых признаков, то число нулевых строк в матрице смежности как раз и будет характеризовать сложность модели.

\bibliographystyle{unsrt}
\extrasrussian
\bibliography{bibliography}

\end{document}
