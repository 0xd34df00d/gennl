\documentclass[12pt,a4paper]{amsart}
\usepackage{a4wide}
\usepackage[utf8]{inputenc}
\usepackage[T2A]{fontenc}
\usepackage{graphics,graphicx,epsfig}
\usepackage{amssymb,amsfonts,amsthm,amsmath,mathtext,cite,enumerate,float}
\usepackage[english,russian]{babel}
\usepackage[all]{xy}
\usepackage{morefloats}
\usepackage{pgf}
\usepackage[outputdir={docgraphs/}]{dot2texi}
\usepackage{tikz}
\usepackage{scalefnt}
\usetikzlibrary{shapes,arrows}
\usetikzlibrary{decorations.pathmorphing}

% Comment the following block when compiling this .tex with a saner compiler than texlive.
\makeatletter
\def\@settitle{\begin{center}%
    \baselineskip14\p@\relax
    \bfseries
    \@title
  \end{center}%
}
\makeatother

\begin{document}
% Comment the following block when compiling this .tex with a saner compiler than texlive.
\pagestyle{plain}

\title{Алгоритмы порождения существенно нелинейных моделей}
\author{Г.\,И.~Рудой}
\address{Московский физико-технический институт, ФУПМ, каф. <<Интеллектуальные системы>>}
\thanks{Научный руководитель В.\,В.~Стрижов}

\begin{abstract}
  В работе исследуются алгоритмы порождения и отбора существенно нелинейных моделей (символьная регрессия). Полученные модели применяются для восстановления регрессионной зависимости переменной от данных числовых рядов. Описывается представление моделей в виде матриц смежности. В вычислительном эксперименте приводятся результаты для задачи моделирования волатильности опционов.
\end{abstract}

\maketitle

\section{Введение}

В ряде приложений возникает задача восстановления регрессии по набору
измеренных данных с условием возможности проинтерпетировать полученные данные
экспертом, например <список ссылок на работы по конкретным приложениям>.

Одним из методов, позволяющих получать интерпретируемые модели, является
символьная регрессия --- процесс, в котором измеренные данные приближаются
некоторой математической формулой, например $ \sin x^2 + 2x $ или
$\log x - \frac{e^x}{x} $. Одна из возможных реализаций этого процесса
предложена John Koza \cite{Koza1998GP} \cite{Koza1998Intro}, использовавшим
эволюционные алгоритмы для реализации символьной регрессии. Ivan Zelinka
предложил дальнейшее развитие этой идеи \cite{Zelinka2008}, получившее
название Analytic Programming.

В подобных случаях алгоритм построения требуемой математической формулы
выглядит следующим образом: дан набор примитивных функций, из которых можно
строить различные формулы (например, степенная функция, $+$, $\sin$, $\tan$).
Начальный набор формул строится либо произвольным образом, либо на базе
некоторых предположений эксперта. Затем на каждом шаге производится оценка
каждой из формул (например, считается функционал SSE). На базе этой оценки
у некоторой части формул случайным образом заменяется одна элементарная функция на другую (например, $\sin$ на $\cos$ или $+$ на $\times$), а у некоторой
другой части происходит взаимный попарный обмен подвыражениями в формулах.

Среди возможных путей улучшения качества символьной регрессии --- анализ
информативности различных признаков. Например, в ходе работы эволюционного
алгоритма можно выявлять, какие из параметров слабо влияют на качество
получающейся формулы, и либо убирать их совсем, либо обеспечивать
неслучайность замены элементарных функций или обмена поддеревьев с целью
замены этих параметров на другие в предположении, что они, возможно,
окажутся более информативными.

Другим вопросом, возникающим при применении подобных эволюционных алгоритмов,
является их принципиальная теоретическая корректность: способен ли вообще
такой алгоритм породить искомую формулу.

\section{Постановка задачи}

\subsection{Теоретическая часть}

Пусть дано множество примитивных функций $G = { g_1, .., g_n }$. Требуется:

\begin{itemize}
  \item Построить алгоритм, за конечное время порождающий любую конечную функцию,
	являющуюся суперпозицией данных примитивных функций.
  \item Указать способ проверки изоморфности двух функций.
\end{itemize}

\subsection{Алгоритмическая часть}

Пусть дан набор $(x_i, y_i) \mid i \in {1,..,N}, x_i \in R^n, y_i \in R$.
Требуется построить аналитическую функцию $f : R^n \rightarrow R$, доставляющую
минимум некоторому функционалу ошибки.

\section{Пути решения задачи: теоретическая часть}


\bibliographystyle{unsrt}
\extrasrussian
\bibliography{bibliography}

\end{document}
